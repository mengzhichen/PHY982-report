\section{Choice of beam energies and potentials} \label{part1}
	The $^{12}$C(d, p)$^{13}$C reaction is discussed in this work. 
	The Coulomb barrier between $^{12}$C and deutron is about 2.03 MeV.
	Then, two deuteron beam energies are studied, one is 2.84 MeV near the Coulomb barrier; another one is 4.51 MeV which is about 2 times higher than the barrier .
	
	At both of the experimental energies chosen, the reaction is more accurately modeled as a compound reaction, as mentioned in \cite{PhysRev.101.209}.
	Therefore our calculations using FRESCO may not yield satisfying agreement with experiment.
	
	Optical potentials are needed that described the incoming and outgoing distorted waves.  
	These are interactions between: $^{12}$C, d \cite{PhysRevC.73.054605} and $^{12}$C, p\cite{PTCOG}.
	For the deuteron wavefunction, the binding for the proton and neutron was described by a simple gaussian potential 
	\begin{equation}
		V_{np}(r)=-72.15e^{-(r/1.484)^2}.
	\end{equation}
	scaled to reproduce a bound state at 2.2 MeV.
	
	The neutron that is transferred in the reaction is expected to occupy a $1p_{1/2}$ orbit with an experimental single-particle binding energy of 4.946 MeV. 
	The FRESCO calculation dynamically adjusts the Woods-Saxon depth for $^{13}$C to reproduce this energy.
	
	