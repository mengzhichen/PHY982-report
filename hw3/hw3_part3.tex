\section{Results of prior-form DWBA calculations}
In a transfer reaction A(d,p)B, the transfer T-matrix has a prior-form formula \cite{thompson2009nuclear}
\begin{equation}\label{eq:priorexact}
T_{prior}=\langle\Psi_2^{(-)}(\vec{r}_2,\vec{R}_2)\left|V_{nA}(r_n)+U_{pA}(r_p)-U_{dA}(R_1)\right|\phi_{np}\chi_{dA}\rangle,
\end{equation}
where $\phi_{np}$ and $\chi_{dA}$ are bound states wave-functions, $\vec{r}_n=\vec{r}_p+\vec{r}_1$, 
and $U_{dA}(R_1)$ is the auxiliary potential we choose. 
Under first-order DWBA, it becomes
\begin{equation}\label{tprior}
T_{prior}=\langle\phi_{nA}\chi_{pB}^{(-)}\left|V_{nA}(r_n)+U_{pA}(r_p)-U_{dA}(R_1)\right|\phi_{np}\chi_{dA}\rangle,
\end{equation}
\par
The differential cross sections of $^{12}$C(d, p)$^{13}$C calculated in both post and prior forms are given in Fig. ??. 
First-order DWBA with finite-range interactions and full complex remnant is used in both types of calculations. 
The convergence of calculation in prior form is checked in the same way as we discussed in Sec. \ref{sec:post}. 
Variables $rnl$ and $center$ are chosen based on FRESCO's recommendations. 
As shown in Fig. ??, the results from post- and prior-form calculations have similar trends and are not far from each other. 
If we calculate without any approximation (Eq. \ref{eq:postexact} and \ref{eq:priorexact}), 
the same results should be obtained from both forms. 
But if only first-order DWBA is considered, we expect that small differences between the two results appear, 
which can be seen in Fig. ??. 
\par
It is worth mentioning that the recommended $rnl$, which represents the non-local range, is larger in prior form (??) than that in post form (??). 
In post form (Eq. \ref{tpost}) $U_{pA}(r_p)$ and $U_{pB}(R_2)$ are close to each other as nuclei A and B are very similar. 
Thus, the operator in Eq. \ref{tpost} is approximately $V_{np}(r_1)$, which has a very short range. 
However, in prior form (Eq. \ref{tprior}) $U_{pA}(r_p)$ and $U_{dA}(R_1)$ cannot cancel each other as the elastic scatterings of deuteron on A and proton on A are very different. 
Thus, the operator in Eq. \ref{tprior} has a longer range which comes from optical potentials $U_{pA}(r_p)$ and $U_{dA}(R_1)$. 
A larger $rnl$ in prior-form calculation also leads to a longer runtime. 