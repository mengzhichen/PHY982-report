\section{Extraction of spectroscopic factor}
As is normal, the spectroscopic factor is extracted by comparing the theory to the data at the first peak in the angular distribution
\cite{PhysRevC.69.064313}, as we expect that the reaction is mostly direct at the forward angle. 
The spectroscopic factors at beam energies 2.84 MeV and 4.51 MeV are given in Table \ref{tab:spec}. 
The angle of first peak $\theta_{p}$ and corresponding differential cross section $\sigma^{\mathrm{DWBA}}$ is given by the first-order DWBA calculations in post form with finite-range interactions and full complex remnant (see Sec. \ref{sec:post}). 
$\sigma^{\mathrm{exp}}$ is obtained by spline interpolation of experimental data. 
The spectroscopic factors we extract are energy-dependent, 
which is expected because Ref. \cite{PhysRev.101.209} shows that compound nucleus formation has a considerable contribution in the reaction mechanism.  
\begin{table}[bt]
	\centering
	\caption{Spectroscopic factors $S$ extracted from $^{12}$C(d, p)$^{13}$C. 
	$\theta_p$ is the angle of the first peak, and $\sigma^{\mathrm{exp}}$ and $\sigma^{\mathrm{DWBA}}$ are corresponding differential cross sections obtained from experimental data and post-form DWBA calculation, respectively. }
	\label{tab:spec}
	\begin{tabular}{ccc}
		\hline
		\hline
		Beam energy (MeV)                  & 2.84 & 4.51 \\
		\hline
		$\theta_p$ (degree)                &  31 & 25 \\
		$\sigma^{\mathrm{exp}}$ (mb/sr)    &  18.49 & 11.57 \\
		$\sigma^{\mathrm{DWBA}}$ (mb/sr)   &  43.14 & 56.40 \\
		$S=\sigma^{\mathrm{exp}}/\sigma^{\mathrm{DWBA}}$ & 0.4286  & 0.2051 \\
		\hline
		\hline
	\end{tabular}
\end{table}