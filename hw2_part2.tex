\section{Fit optical potentials for elastic scattering of a nucleon on $^{208}$Pb target}
In this section we will do $\chi^2$ data fitting to obtain optical potentials 
for the elastic scattering of a proton or neutron on $^{208}$Pb. 
Experimental data are taken from Ref. \cite{MANI1971384_ProtonPb208Data} for proton and 
and \cite{PhysRevC.85.024619_NeutronPb208Data} for neutron. 
The beam energies of proton and neutron are 49.35 MeV and 40.0 MeV, respectively. 
In addition, we choose the optical parameters employed in Sec. \ref{part1} (Ref. \cite{capote2009ripl,koning2003local}) 
as the starting point of our fitting. 
All the results discussed in this section are generated by SFRESCO (Ref. \cite{FRESCO}). 

\subsection{Fit process and results}
All the intermediate and final results of our fitting are summarized in Table \ref{proton_table} (for proton) and Table \ref{neutron_table} (for neutron). 
For simplicity, we neglect spin-orbit components at the beginning, and only volume real and volume imaginary components are included. 
Keeping the volume imaginary component unchanged, we fit the volume real part (index 2 in Table \ref{proton_table} and \ref{neutron_table}), 
which cannot reduce $\chi^2/N$ to a low value. 
Then, we vary both volume real and imaginary parts simultaneously (index 3 in Table \ref{proton_table} and \ref{neutron_table}) and 
achieve a significant improvement.  
However, for proton the diffuseness parameter av becomes too small to be considered physical. 
\par
Another choice of optical potential is to use a surface imaginary component instead of a volume one. 
Starting from parametrization of index 4 in Table \ref{proton_table} and \ref{neutron_table}, we fit the volume real part and surface imaginary part 
(index 5 in Table \ref{proton_table} and \ref{neutron_table}). 
For both proton and neutron, a surface imaginary component gives a lower $\chi^2/N$, 
and thus we will add spin-orbit terms onto this optical potential form. 
\par
As shown in the last four lines in Table \ref{proton_table} and \ref{neutron_table}, 
we gradually add and fit different parameters in the spin-orbit term.  
As for proton, the description can hardly be improved by only adding a real spin-orbit potential (index 6 in Table \ref{proton_table}).  
If we only add a imaginary spin-orbit component, the diffuseness parameter ``awso" will become negative and thus unphysical 
(index 8 in Table \ref{proton_table}), so we abandon this set of parameters. 
If both real and imaginary components of spin-orbit potential are included, $\chi^2/N$ will slightly decrease from 4.860 to 3.611,
(index 9 in Table \ref{proton_table}), which indicates that the spin-orbit term is not very important 
for the description of elastic scattering of a proton on $^{208}$Pb. 
\par
As for neutron, the radius parameter ``rvso'' will become negative and unphysical when only a real spin-orbit component is included 
(index 6 in Table \ref{neutron_table}), so we abandon this set of parameters. 
By only adding an imaginary spin-orbit component, $\chi^2/N$ is slightly lowered from 4.638 to 3.310 
(index 8 in Table \ref{neutron_table}). 
When both real and imaginary spin-orbit parts are varied, $\chi^2/N$ can be even lower (index 9 in Table \ref{neutron_table}). 
However, in this case the diffuseness ``avso" will become less than radius ``rvso", which is unphysical and unacceptable. 
As no significant improvement is seen from adding spin-orbit term, we conclude that 
the spin-orbit term is not very important for the description of elastic scattering of a neutron on $^{208}$Pb. 

\begin{landscape}
	\begin{table}[t]
		\centering
		\caption{Fitting results of scattering of a proton on $^{208}$Pb. 
		The unit of Vv, Wv, Ws, Vso and Wso is MeV, and the unit of rv, av, rwv, awv, rws, aws, rvso, avso, rwso and awso is fm. 
	    Unphysical values are highlighted in red. }
		\label{proton_table}
		\footnotesize
		\begin{tabular}{cccccccccccccccccc}
			\hline
			\hline
			Index & Fit from & Vv  & rv  & av  & Wv & rwv & awv  & Ws & rws & aws & Vso & rvso & avso & Wso & rwso & awso & $\chi^2/N$ \\
			\hline
			1     & -          & 67.2     & 1.244   & 0.646    & 16.6     & 1.244    & 0.646    & -        & -        & -        & -         & -         & -         & -         & -         & -        & 78.285   \\
			2     & 1          & 46.1     & 1.262   & 0.574    & 16.6     & 1.244    & 0.646    & -        & -        & -        & -         & -         & -         & -         & -         & -        & 27.134   \\
			3     & 2          & 44.6     & 1.254   & 1.87E-04 & 4.89     & 1.726    & 0.171    & -        & -        & -        & -         & -         & -         & -         & -         & -        & 5.479    \\
			4     & -          & 46.1     & 1.262   & 0.574    & -        & -        & -        & 19.5     & 1.246    & 0.615    & -         & -         & -         & -         & -         & -        & 11.335   \\
			5     & 4          & 45.3     & 1.229   & 0.528    & -        & -        & -        & 13.2     & 1.295    & 0.713    & -         & -         & -         & -         & -         & -        & 4.860    \\
			6     & 5          & 45.2     & 1.232   & 0.489    & -        & -        & -        & 12.7     & 1.295    & 0.732    & 0.14     & 1.07     & 0.55     & -         & -         & -        & 4.838    \\
			7     & -          & 45.2     & 1.232   & 0.489    & -        & -        & -        & 12.7     & 1.295    & 0.732    & -         & -         & -         & -3.1    & 1.08      & 0.57     & 12.026   \\
			8     & 7          & 45.1     & 1.242   & 0.507    & -        & -        & -        & 12.0     & 1.279    & 0.748    & -         & -         & -         & -2.1     & 0.66     & \color{red}{-0.028}   & 4.068    \\
			9     & 7          & 45.2     & 1.276   & 0.540    & -        & -        & -        & 15.7     & 1.239    & 0.710    & -6.9     & 0.84     & 0.55     & -8.0     & 1.03      & 0.53     & 3.661   \\
			\hline
			\hline
		\end{tabular}
	\end{table}

\begin{table}[b]
	\centering
	\caption{Fitting results of scattering of a neutron on $^{208}$Pb. 
			The unit of Vv, Wv, Ws, Vso and Wso is MeV, and the unit of rv, av, rwv, awv, rws, aws, rvso, avso, rwso and awso is fm. 
			Unphysical values are highlighted in red. }
	\label{neutron_table}
	\footnotesize
	\begin{tabular}{cccccccccccccccccc}
		\hline
		\hline
		index & Fit from & Vv   & rv    & av    & Wv    & rwv   & awv   & Ws   & rws   & aws   & Vso & rvso  & avso & Wso   & rwso & aso  & $\chi^2/N$ \\
		\hline
		1     & -        & 50.6 & 1.244 & 0.646 & 15.6  & 1.244 & 0.646 & -    & -     & -     & -   & -     & -    & -     & -    & -    & 261.394    \\
		2     & 1        & 40.3 & 1.245 & 0.903 & 15.6  & 1.244 & 0.646 & -    & -     & -     & -   & -     & -    & -     & -    & -    & 46.553     \\
		3     & 2        & 41.9 & 1.154 & 0.762 & 6.598 & 1.383 & 0.824 & -    & -     & -     & -   & -     & -    & -     & -    & -    & 7.900      \\
		4     & -        & 40.3 & 1.245 & 0.903 & -     & -     & -     & 13.8 & 1.246 & 0.510 & -   & -     & -    & -     & -    & -    & 98.423     \\
		5     & 4        & 36.4 & 1.269 & 0.585 & -     & -     & -     & 14.4 & 1.097 & 0.500 & -   & -     & -    & -     & -    & -    & 4.638      \\
		6     & 5        & 36.1 & 1.277 & 0.578 & -     & -     & -     & 15.7 & 1.108 & 0.498 & 8.3 & \color{red}{-4.3} & 3.2 & -     & -    & -    & 3.084      \\
		7     & -        & 36.4 & 1.269 & 0.585 & -     & -     & -     & 14.4 & 1.097 & 0.500 & -   & -     & -    & -3.1 & 1.08 & 0.57 & 35.163     \\
		8     & 7        & 36.4 & 1.273 & 0.593 & -     & -     & -     & 15.2 & 1.109 & 0.519 & -   & -     & -    & -3.1 & 1.09 & 0.43 & 3.310      \\
		9     & 8        & 35.8 & 1.266 & 0.592 & -     & -     & -     & 14.9 & 1.107 & 0.513 & 4.9 & \color{red}{1.03}  & \color{red}{1.23} & -3.0 & 1.06 & 0.81 & 2.258      \\
		\hline
		\hline
	\end{tabular}
\end{table}
\end{landscape}

\subsection{Sensitivity of final parameters to the initialization}
Still empty. 
